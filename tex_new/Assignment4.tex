\documentclass[11pt,a4paper,titlepage]{article}
\usepackage[a4paper]{geometry}
\usepackage[utf8]{inputenc}
\usepackage[english]{babel}
\usepackage{lipsum}

\usepackage{amsmath, amssymb, amsfonts, amsthm, fouriernc, mathtools}
% mathtools for: Aboxed (put box on last equation in align envirenment)
\usepackage{microtype} %improves the spacing between words and letters
\usepackage{fouriernc}
\usepackage{graphicx}
\graphicspath{ {./pics/} {./eps/}}
\usepackage{epsfig}
\usepackage{epstopdf}


%%%%%%%%%%%%%%%%%%%%%%%%%%%%%%%%%%%%%%%%%%%%%%%%%%
%% COLOR DEFINITIONS
%%%%%%%%%%%%%%%%%%%%%%%%%%%%%%%%%%%%%%%%%%%%%%%%%%
\usepackage[svgnames]{xcolor} % Enabling mixing colors and color's call by 'svgnames'
%%%%%%%%%%%%%%%%%%%%%%%%%%%%%%%%%%%%%%%%%%%%%%%%%%
\definecolor{MyColor1}{rgb}{0.2,0.4,0.6} %mix personal color
\newcommand{\textb}{\color{Black} \usefont{OT1}{lmss}{m}{n}}
\newcommand{\blue}{\color{MyColor1} \usefont{OT1}{lmss}{m}{n}}
\newcommand{\blueb}{\color{MyColor1} \usefont{OT1}{lmss}{b}{n}}
\newcommand{\red}{\color{LightCoral} \usefont{OT1}{lmss}{m}{n}}
\newcommand{\green}{\color{Turquoise} \usefont{OT1}{lmss}{m}{n}}
%%%%%%%%%%%%%%%%%%%%%%%%%%%%%%%%%%%%%%%%%%%%%%%%%%




%%%%%%%%%%%%%%%%%%%%%%%%%%%%%%%%%%%%%%%%%%%%%%%%%%
%% FONTS AND COLORS
%%%%%%%%%%%%%%%%%%%%%%%%%%%%%%%%%%%%%%%%%%%%%%%%%%
%    SECTIONS
%%%%%%%%%%%%%%%%%%%%%%%%%%%%%%%%%%%%%%%%%%%%%%%%%%
\usepackage{titlesec}
\usepackage{sectsty}
%%%%%%%%%%%%%%%%%%%%%%%%
%set section/subsections HEADINGS font and color
\sectionfont{\color{MyColor1}}  % sets colour of sections
\subsectionfont{\color{MyColor1}}  % sets colour of sections

%set section enumerator to arabic number (see footnotes markings alternatives)
\renewcommand\thesection{\arabic{section}.} %define sections numbering
\renewcommand\thesubsection{\thesection\arabic{subsection}} %subsec.num.

%define new section style
\newcommand{\mysection}{
\titleformat{\section} [runin] {\usefont{OT1}{lmss}{b}{n}\color{MyColor1}} 
{\thesection} {3pt} {} } 

%%%%%%%%%%%%%%%%%%%%%%%%%%%%%%%%%%%%%%%%%%%%%%%%%%
%		CAPTIONS
%%%%%%%%%%%%%%%%%%%%%%%%%%%%%%%%%%%%%%%%%%%%%%%%%%
\usepackage{caption}
\usepackage{subcaption}
%%%%%%%%%%%%%%%%%%%%%%%%
\captionsetup[figure]{labelfont={color=Turquoise}}

%%%%%%%%%%%%%%%%%%%%%%%%%%%%%%%%%%%%%%%%%%%%%%%%%%
%		!!!EQUATION (ARRAY) --> USING ALIGN INSTEAD
%%%%%%%%%%%%%%%%%%%%%%%%%%%%%%%%%%%%%%%%%%%%%%%%%%
%using amsmath package to redefine eq. numeration (1.1, 1.2, ...) 
%%%%%%%%%%%%%%%%%%%%%%%%
\renewcommand{\theequation}{\thesection\arabic{equation}}

%set box background to grey in align environment 
\usepackage{etoolbox}% http://ctan.org/pkg/etoolbox
\makeatletter
\patchcmd{\@Aboxed}{\boxed{#1#2}}{\colorbox{black!15}{$#1#2$}}{}{}%
\patchcmd{\@boxed}{\boxed{#1#2}}{\colorbox{black!15}{$#1#2$}}{}{}%
\makeatother
%%%%%%%%%%%%%%%%%%%%%%%%%%%%%%%%%%%%%%%%%%%%%%%%%%




%%%%%%%%%%%%%%%%%%%%%%%%%%%%%%%%%%%%%%%%%%%%%%%%%%
%% DESIGN CIRCUITS
%%%%%%%%%%%%%%%%%%%%%%%%%%%%%%%%%%%%%%%%%%%%%%%%%%
\usepackage[siunitx, american, smartlabels, cute inductors, europeanvoltages]{circuitikz}
%%%%%%%%%%%%%%%%%%%%%%%%%%%%%%%%%%%%%%%%%%%%%%%%%%



\makeatletter
\let\reftagform@=\tagform@
\def\tagform@#1{\maketag@@@{(\ignorespaces\textcolor{red}{#1}\unskip\@@italiccorr)}}
\renewcommand{\eqref}[1]{\textup{\reftagform@{\ref{#1}}}}
\makeatother
\usepackage{hyperref}
\hypersetup{colorlinks=False}

%%%%%%%%%%%%%%%%%%%%%%%%%%%%%%%%%%%%%%%%%%%%%%%%%%
%% PREPARE TITLE
%%%%%%%%%%%%%%%%%%%%%%%%%%%%%%%%%%%%%%%%%%%%%%%%%%
\title{\blueb Scientific Experimentation and Evaluation \\
\blueb Assignment 4 \\ Task 3: Calibrating an Optical Tracking System}
\author{Sushma Devaramani}
\date{\today}
%%%%%%%%%%%%%%%%%%%%%%%%%%%%%%%%%%%%%%%%%%%%%%%%%%



\begin{document}
\maketitle
\tableofcontents
\newpage
\section{Camera calibration setup}
\begin{enumerate}
\item The camera calibration process is done on a checkerboard (calibration tool).
\item Installed the \textit{MATLAB support package for USB webcam} for camera calibration.
\item The camera calibration toolbox for MatLAB is downloaded.
\item The webcam (Microsoft Lifecam) is connected to the laptop. The connectivity is checked by running \textit{webcam list} command in MATLAB, which shows the list of all webcams connected to PC.
\item The webcam is mounted on a static paper punch, while capturing the images.
\item The Auto-focus is turned \textit{off} by running the following command in MATLAB, \\ 
\item The images of checkerboard are captured from different perspectives (different position and orientation in the camera's field of view). A total of 20 images are captured and saved for the calibration process.
\end{enumerate}

\section{Estimation of number of images}
\begin{itemize}
\item Since the object in the world is 3-dimensional and the captured images are 2D (co-planar), we loose one dimension. In order to regain the actual dimensionality during the calibration, various number of images are captured in different positions and orientations. 
\item As a rule of thumb, 20-40 image pairs is quite enough for the process. For our calibration we have captured 20 images in different position and orientations. 
\item The idea behind the capturing enough images is to cover the field of view, and to have a good distribution of 3D orientations of the board. 
\end{itemize}


\section{Description of the camera parameters}
After the calibration, the list of parameters obtained are of two types,
\begin{enumerate}
\item Intrinsic parameters \begin{itemize}
\item \textbf{Focal length:} Distance between the lens and the image sensor when the object is in focus, usually stated in millimeters. The focal length in pixels is stored in the 2x1 vector $fc$. \\ \textit{Calibration result:} $[648.5545, 648.4]$
\item \textbf{Principal point:} A point at the intersection of the optical axis and the image plane. The principal point coordinates are stored in the 2x1 vector $cc$. \\ \textit{Calibration result:} $415.7172$
\item \textbf{Skew coefficient:} Defines the angle between x and y pixels axes and is stored in $alpha_c$ \\ \textit{Calibration result:} $0$
\item \textbf{Radial Distortions:} Symmetric distortion caused by the lens due to imperfections in curvature when the lens was ground. \\ \textit{Calibration result:} $[-0.0187, 0.118]$
\item The image distortion coefficients (radial and tangential distortions) are stored in the 5x1 vector $kc$. \\ \textit{Calibration result:} Tangential distortion = $[0,0]$
\end{itemize}
\item Extrinsic parameters \begin{itemize}
\item \textbf{Rotational Vectors:} Vector describing the rotation of camera plane with respect to the world frame. \\ \textit{Calibration result:} $20x3$ double.
\item \textbf{Translational vectors:} Vector describing the translation of camera plane with respect to the world frame. \\ \textit{Calibration result:} $20x3$ double.
\end{itemize}
\end{enumerate}

\section{Possible problems or error sources that can disturb the calibration process}
\begin{enumerate}
\item If the naming conventions of the images are different, then the calibration process gives error.
\item The computer running the calibration toolbox must have enough RAM(128 MB or less). If this condition is not met, then it give \textit{OUT OF MEMORY} error.
\item Enough number of images must be provided to get better calibration results.
\item The images captured with same orientations will not provide enough inputs for the toolbox to calibrate efficiently.
\end{enumerate}

\section{Describe the images poses used for calibration and report the found camera parameters including any error estimates (where applicable)}

\textcolor{red}{Yet to be written}

\section{Arguments for which camera model best fits} 
	If the lens distortions are really too severe (for fish-eye lenses for example), the simple guiding tool based on a single distortion coefficient kc may not be sufficient to provide good enough initial guesses for the corner locations. then corner extraction has to be done manually.



\end{document}